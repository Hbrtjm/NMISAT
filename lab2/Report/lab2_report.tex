\documentclass{article}
\usepackage{graphicx} % Required for inserting images

\title{MOWNIT laboratorium 2}
\author{Hubert Miklas}
\date{Marzec 2025}

\begin{document}

\maketitle



\section{Wstęp}

Laboratorium jest ciągiem dalszym laboratorium 1 które skupia się na arytmetyce komputerowej.

\section{Zadanie 1.}

Celem zadania jest zaimplementowanie algorytmu obliczającego funkcję wykładniczą $e^x$ za pomocą nieskończonego szeregu Maclaurina. Szereg ten ma postać:
\[
e^x = 1 + x + \frac{x^2}{2!} + \frac{x^3}{3!} + \cdots
\]
W tym zadaniu rozważono kilka aspektów implementacji algorytmu, w tym kryterium zakończenia obliczeń, porównanie wyników z funkcją $exp(x)$ oraz pytanie o dokładność dla wartości $x < 0$.


\subsection{Kryterium zakończenia obliczeń}

Aby zakończyć obliczenia, przyjęto, że obliczenia zostaną przerwane, gdy wartość kolejnego składnika szeregu osiągnie wartość mniejszą niż zadana dokładność $\epsilon$. W kodzie implementującym obliczenia, wartość ta została ustawiona na $1 \times 10^{-15}$.

\newpage

\subsection{Kod implementujący obliczenia}

\begin{verbatim}
from math import exp

def facts(n):
    if n == 0:
        return [1]
    results = [1]
    fact = 1
    for i in range(1, n):
        fact *= i
        results.append(fact)
    return results

def maclaurin_exp(x, epsilon=1e-15):
    fact_size = 10
    acc = 1
    fact = facts(fact_size)
    i = 0
    result = 0
    if x == 0:
        return 1
    while abs(acc / fact[i]) > epsilon:
        if i + 1 == fact_size:
            fact_size *= 2
            fact = facts(fact_size)
        result += acc / fact[i]
        acc *= x
        i += 1
    return result

def maclaurin_exp_negative(x, epsilon=1e-15):
    return 1 / maclaurin_exp(-x, epsilon)

def horner_maclaurin_exp(x, n=100, epsilon=1e-15):
    fact = facts(n)
    result = 1 / fact[n - 1]  # Initialize last term
    for i in range(n - 2, -1, -1):
        result = 1 / fact[i] + x * result
    return result

\end{verbatim}

\subsection{Wyniki obliczeń}

Poniżej przedstawiono wyniki obliczeń dla różnych wartości $x$. Dla każdej wartości $x$ obliczono wartość funkcji wykładniczej zarówno za pomocą zaimplementowanego algorytmu Maclaurina, jak i funkcji wbudowanej w bibliotekę Python \texttt{exp()}. Porównano również wyniki obliczeń za pomocą algorytmu Hornera.

\begin{center}
\begin{tabular}{|c|c|c|c|}
    \hline
    $x$ & Moje obliczenia $e^x$ & Biblioteka Python $\exp(x)$ & Błąd \\
    \hline
    -10 & $4.540\times10^{-5}$ & $4.540\times10^{-5}$ & $6.776\times10^{-21}$ \\
    -5  & $0.00674$ & $0.00674$ & $1.735\times10^{-18}$ \\
    -1  & $0.368$ & $0.368$ & $-5.551\times10^{-17}$ \\
    1   & $2.718$ & $2.718$ & $4.441\times10^{-16}$ \\
    5   & $148.413$ & $148.413$ & $-2.842\times10^{-14}$ \\
    10  & $22026.466$ & $22026.466$ & $-7.276\times10^{-12}$ \\
    \hline
\end{tabular}
\end{center}


\subsection{Dokładność dla wartości $x < 0$}

Przeprowadzono obliczenia dla $x < 0$ przy pomocy standardowego algorytmu oraz zmodyfikowanej wersji algorytmu, w której obliczenia są wykonywane za pomocą $e^{-x}$, a następnie wynik jest odwrotnością obliczonej wartości. Widać, że uzyskane wyniki są zgodne z wartościami obliczonymi przy użyciu funkcji $\exp(x)$ z biblioteki Python. Błędy dla wartości $x < 0$ są również minimalne i mieszczą się w granicach dokładności $\epsilon$.

\subsection{Podsumowanie i wnioski na temat lepszego obliczania szeregu}

Zaproponowany algorytm do obliczania funkcji wykładniczej \(e^x\) opiera się na klasycznym szeregu Taylora:
\[
e^x = \sum_{n=0}^{\infty} \frac{x^n}{n!}.
\]
Dla wartości \(x<0\) bezpośrednie sumowanie szeregu może prowadzić do problemów numerycznych związanych z sumowaniem wyrazów o zmiennych znakach, co zwiększa ryzyko utraty precyzji (tzw. \emph{catastrophic cancellation}). Aby uzyskać dokładniejsze wyniki, zamiast bezpośrednio sumować wyrazy szeregu, wykorzystuje się tożsamość:
\[
e^x = \frac{1}{e^{-x}},
\]
co pozwala na obliczenie szeregu dla \(e^{-x}\) (przy \(x>0\)), w którym wszystkie składniki są dodatnie. Następnie, poprzez wyliczenie odwrotności, uzyskuje się wartość \(e^x\). Takie przegrupowanie składników zmniejsza błędy zaokrągleń i zwiększa stabilność numeryczną obliczeń \cite{Higham2002,Goldberg1991}.

\section{Zadanie 2}

Celem zadania jest porównanie dokładności dwóch matematycznie równoważnych wyrażeń:
\[
x^2 - y^2 \quad \text{oraz} \quad (x - y)(x + y)
\]
w kontekście obliczeń arytmetyki zmiennoprzecinkowej. Oba wyrażenia są matematycznie ekwiwalentne, jednak przy obliczeniach komputerowych mogą wystąpić różnice wynikające z procesów zaokrągleń. 

\subsection{Porównanie wyrażeń}

Obliczenia przeprowadzone dla różnych wartości zmiennych \(x\) i \(y\) pokazują, że bezpośrednie obliczenie \(x^2 - y^2\) może prowadzić do utraty precyzji, szczególnie gdy \(x\) i \(y\) są bardzo zbliżone, ponieważ następuje odejmowanie dwóch dużych liczb. Natomiast wyrażenie \((x-y)(x+y)\) najpierw wykonuje odejmowanie, a dopiero potem mnożenie, co redukuje wpływ błędów zaokrągleń i czyni to wyrażenie numerycznie bardziej stabilnym \cite{Goldberg1991}.


\subsection{Kod implementujący obliczenia}

\begin{verbatim}
def expression_1(x,y):
    return x**2 - y**2

def expression_2(x,y):
    return (x+y) * (x-y)

test_values = [
    (1e-14, 1e-14),
    (1e-14, -1e-14),
    (-1e-14, 1e-14),
    (-1e-14, -1e-14),
    (1.1e-14, 1e-14),
    (1e-14, 1.1e-14),
]

for x,y in test_values:
    expression_1_value = expression_1(x,y)
    expression_2_value = expression_2(x,y)
    print(f"Value from the x^2 - y^2 {expression_1_value}")
    print(f"Value from the (x + y)(x - y) {expression_2_value}")
    print(f"Difference between expr_1 and expr_2: {expression_1_value - expression_2_value}")
\end{verbatim}


\subsection{Wyniki obliczeń}

Poniższa tabela przedstawia wyniki obliczeń dla przykładowych wartości zmiennych \(x\) i \(y\):

\begin{center}
\begin{tabular}{|c|c|c|}
    \hline
    \(x\) i \(y\) & Wyrażenie \(x^2 - y^2\) & Wyrażenie \((x+y)(x-y)\) \\
    \hline
    \((1\cdot10^{-14},\, 1\cdot10^{-14})\) & 0.0 & 0.0   \\
    \((1\cdot10^{-14},\, -1\cdot10^{-14})\) & 0.0 & 0.0  \\
    \((-1\cdot10^{-14},\, 1\cdot10^{-14})\) & 0.0 & -0.0  \\
    \((-1\cdot10^{-14},\, -1\cdot10^{-14})\) & 0.0 & -0.0 \\
    \((1.1\cdot10^{-14},\, 1\cdot10^{-14})\) & \(2.1000000000000007 \times 10^{-29}\) & \(2.100000000000001 \times 10^{-29}\)  \\
    \((1\cdot10^{-14},\, 1.1\cdot10^{-14})\) & \(-2.1000000000000007 \times 10^{-29}\) & \(-2.100000000000001 \times 10^{-29}\) \\
    \hline
\end{tabular}
\end{center}

\subsection{Analiza wyników}

Z wyników obliczeń wynika, że przy bardzo małych wartościach zmiennych (rzędu \(10^{-14}\)) oba wyrażenia dają identyczne wyniki równe zeru w kontekście arytmetyki zmiennoprzecinkowej. Różnice w wynikach mogą pojawiać się przy bardziej zróżnicowanych wartościach zmiennych. W takich przypadkach, wyrażenie \((x-y)(x+y)\) okazuje się być bardziej stabilne numerycznie, gdyż minimalizuje efekty błędów zaokrągleń wynikających z odejmowania dużych, niemal równych wartości w wyrażeniu \(x^2-y^2\) \cite{Rycerz}.

\subsection{Podsumowanie}

W arytmetyce zmiennoprzecinkowej wyrażenie \((x-y)(x+y)\) jest zwykle obliczane z większą dokładnością niż \(x^2-y^2\), szczególnie gdy wartości \(x\) i \(y\) są bardzo zbliżone. Jest to związane z mniejszym ryzykiem utraty precyzji przy odejmowaniu.

\section{Zadanie 3}

Celem zadania jest analiza dokładności obliczeń wyróżnika równania kwadratowego w zmiennoprzecinkowej arytmetyce przy użyciu znormalizowanego systemu zmiennoprzecinkowego. Rozważamy równanie kwadratowe w postaci:
\[
ax^2 + bx + c = 0,
\]
gdzie \(a = 1.22\), \(b = 3.34\) oraz \(c = 2.28\). Obliczenia wykonujemy w systemie zmiennoprzecinkowym o podstawie \(\beta = 10\) i dokładności \(p = 3\) (czyli z trzema cyframi znaczącymi).

\subsection{Podstawowe obliczenia}

Wyróżnik równania kwadratowego wyraża się wzorem:
\[
\Delta = b^2 - 4ac.
\]
Zadanie składa się z następujących kroków:
\begin{enumerate}
    \item \textbf{(a)} Obliczenie wartości wyróżnika \(\Delta\) w znormalizowanym systemie zmiennoprzecinkowym (z uwzględnieniem ograniczonej precyzji),
    \item \textbf{(b)} Wyznaczenie dokładnej wartości wyróżnika w rzeczywistej arytmetyce,
    \item \textbf{(c)} Oszacowanie względnego błędu obliczonej wartości wyróżnika.
\end{enumerate}

Ze względu na ograniczoną precyzję (trzy cyfry znaczące), operacje arytmetyczne wykonywane na liczbach mogą skutkować znaczącą utratą precyzji, szczególnie gdy składniki \(b^2\) i \(4ac\) są do siebie bardzo zbliżone. W takim przypadku nawet niewielkie błędy zaokrągleń mogą wpłynąć na końcowy wynik, co jest szczególnie istotne przy rozwiązywaniu równań kwadratowych.

\subsection{Kod implementujący obliczenia}

\begin{verbatim}
    class Quadratic:
    def __init__(self,a,b,c):
        self.a = a
        self.b = b
        self.c = c

    def evaluate_simple(self,x):
        return x**2 * self.a + self.b * x + self.c
    
    def evaluate_horner(self,x):
        return self.c + x * ( self.b + x * self.a ) 
    
    def delta(self):
        return self.b ** 2 - 4*self.a*self.c
    

def main():
    a = 1.22
    b = 3.34
    c = 2.28
    q = Quadratic(a,b,c)
    delta = q.delta()
    print(f"The value of delta is {delta}")
    real_delta = 0.0292
    print(f"The real value of delta is {real_delta}")
    print(f"Relatice difference  between the real_delta and delta {abs(real_delta-delta)/real_delta}")

if __name__ == "__main__":
    main()
\end{verbatim}

\newpage

\subsection{Analiza wyników}

Analiza wyników obliczeń pozwala zauważyć, że:
\begin{itemize}
    \item Obliczona wartość wyróżnika w systemie zmiennoprzecinkowym z ograniczoną precyzją może znacznie odbiegać od dokładnej wartości obliczonej w arytmetyce rzeczywistej.
    \item W przypadku, gdy \(b^2\) oraz \(4ac\) są do siebie bardzo zbliżone, względny błąd obliczeń wyróżnika staje się bardzo duży.
\end{itemize}

\subsection{Podsumowanie}

Przeprowadzona analiza pokazuje, że w systemach o ograniczonej precyzji obliczeniowej należy szczególnie uważać przy operacjach, w których następuje odejmowanie dwóch niemal równych liczb. Zarówno przy obliczaniu szeregu wykładniczego dla \(x<0\), jak i przy wyliczaniu wyróżnika równania kwadratowego, zastosowanie metod minimalizujących utratę precyzji (np. przegrupowanie składników lub odpowiednia reformulacja wyrażeń) może znacząco poprawić wyniki obliczeń \cite{Higham2002,Goldberg1991}.

\bibliographystyle{plain}
\bibliography{references}


\end{document}

