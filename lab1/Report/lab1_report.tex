\documentclass{article}
\usepackage{graphicx}
\usepackage{caption}
\usepackage{booktabs}
\usepackage{amssymb}
\usepackage{amsmath}
\usepackage[dvipsnames]{xcolor}
\usepackage[T1]{fontenc}
\usepackage[utf8]{inputenc}
\usepackage{subcaption} 
\usepackage{float}
\usepackage{geometry}
\geometry{margin=1in}
\usepackage{babel}
\usepackage{animate}
\usepackage{hyphenat}

\title{MOWNiT Laboratorium 1\\Precyzja zmiennoprzecinkowa}
\author{Hubert Miklas}
\date{11-03-2025}

\begin{document}

\maketitle

\section{Wprowadzenie}

Laboratorium polega na wprowadzeniu do precyzji obliczeń w operacjach zmiennoprzecinkowych w komputerze. Wszystkie zadania zostały wykonane w języku Python 3.10.12.

\section{Zadanie 1}

Szukamy precyzji komputerowej $\epsilon$, czyli najmniejszej liczby, dla której komputer uznaje $1 + \epsilon > 1$ za prawdziwe. 

\subsection{Kod implementujący obliczenia}

\texttt{\\
epsilon = 1\\
const = 1\\
while const + epsilon > const:\\
\quad epsilon /= 2\\
epsilon *= 2\\
print(epsilon)\\
}

To daje wynik $\epsilon \approx 2.2\times10^{-16}$, co jest oczekiwane dla liczb zmiennoprzecinkowych z podwójną precyzją (float64).


\section{Zadanie 2}

Rozważamy problem ewaluacji funkcji $\sin(x)$, przy czym występuje propagacja błędu danych wejściowych (zakłócenie $h$ w argumencie $x$). Zakładamy $h = 10^{-5}$.

\subsection{Błąd bezwzględny} 
\[
\text{bb} = \left|\sin(x+h) - \sin(x)\right| \approx |\cos(x)|\, h.
\]
Maksymalny błąd bezwzględny wynosi zatem około $10^{-5}$ (dla $|\cos(x)|=1$).

\subsection{Błąd względny}
\[
\text{bw} = \frac{|\sin(x+h) - \sin(x)|}{|\sin(x)|} \approx h\, |\cot(x)|.
\]
Błąd względny rośnie, gdy $\sin(x)$ jest bliskie zeru, czyli dla $x=k\pi$, $k\in\mathbb{Z}$.

\subsection{Uwarunkowanie problemu} 
\[
\text{cond} \approx \left|\frac{x\cos(x)}{\sin(x)}\right| = |x\,\cot(x)|.
\]
Problem jest bardzo czuły dla argumentów $x$ bliskich wielokrotności $\pi$, gdzie $\sin(x) \approx 0$, natomiast lepiej uwarunkowany dla $x\approx \pi/2+k\pi$. W szczególności, dla $x\to 0$ mamy
\[
\lim_{x\to0}\frac{\sin(x)}{x} = 1,
\]
co zapewnia dobre uwarunkowanie w pobliżu zera.

\subsection{Kod implementujący obliczenia}

\begin{verbatim}
import math from sin, cos

def find_machine_epsilon():
    eps = 1.0
    while (1.0 + eps) > 1.0:
        eps /= 2
    return eps * 2

def absolute_error(x, h):
    return abs(sin(x + h) - sin(x))

def relative_error(x, h):
    if sin(x) == 0:
        return float('inf')  # unbounded error
    return abs(h * cos(x) / sin(x))

def condition_number(x):
    if sin(x) == 0:
        return float('inf')
    return abs(x * cos(x) / sin(x))

h = 1e-5  # małe zakłócenie

precision = 4

for x in test_values:
    abs_err = absolute_error(x, h)
    rel_err = relative_error(x, h)
    cond_num = condition_number(x)
    print(f"x = {x}")
    print(f"  Absolute error: {round(abs_err,precision)}")
    print(f"  Relative error: {round(rel_err,precision)}")
    print(f"  Condition number: {round(cond_num,precision)}")
\end{verbatim}

\subsection{Wyniki obliczeń}

Poniżej przedstawiamy wartości błędów i liczby uwarunkowania dla wybranych wartości $x$:

\begin{center}
\begin{tabular}{|c|c|c|c|}
    \hline
    $x$ & Błąd bezwzględny & Błąd względny & Liczba uwarunkowania \\
    \hline
    0.1 & $9.95 \times 10^{-6}$ & $9.967 \times 10^{-5}$ & 0.9967 \\
    0.5 & $8.78 \times 10^{-6}$ & $1.83 \times 10^{-5}$ & 0.9152 \\
    1.0 & $5.40 \times 10^{-6}$ & $6.42 \times 10^{-6}$ & 0.6421 \\
    \hline
\end{tabular}
\end{center}

\section{Zadanie 3}

Funkcję $\sin(x)$ można rozwijać w szereg Maclaurina (szczególny przypadek szeregu Taylora):
\begin{equation}
    \sin(x) = x - \frac{x^3}{3!} + \frac{x^5}{5!} - \frac{x^7}{7!} + \cdots
\end{equation}

\subsection{Przybliżenie pierwszym składnikiem ($\sin(x) \approx x$)}

Przyjmujemy $\hat{y} = x$. Wyznaczamy błędy:
\begin{equation}
    \Delta y = \hat{y} - \sin(x), \quad \Delta x = \arcsin(\hat{y}) - x.
\end{equation}

\subsection{Kod implementujący obliczenia dla obu przypadków}

\begin{verbatim}
from math import sin, asin

def progressive_error_approx1(x):
    return abs(sin(x) - x)

def progressive_error_approx2(x):
    return abs(sin(x) - (x - x**3 / 6))

def backward_error_approx1(x):
    return abs(x - asin(x))

def backward_error_approx2(x):
    return abs(x - asin(x - x**3 / 6))

def underflow_level(beta, L):
    return beta**L

test_values = [0.1, 0.5, 1.0]
precision = 4

print("Approximation 1")

for value in test_values:
    print(f"{value} & {round(backward_error_approx1(value),precision)} & {round(progressive_error_approx1(value),precision)} \")

print("Approximation 2")

for value in test_values:
    print(f"{value} & {round(backward_error_approx2(value),precision)} & {round(progressive_error_approx2(value),precision)} \")

\end{verbatim}

\subsection{Przybliżenie jednym składnikiem ($\sin(x) \approx x$)}

Dla wybranych wartości $x$ otrzymujemy:

\begin{table}[H]
    \centering
    \begin{tabular}{c|cc}
         $x$ & $\Delta y$ & $\Delta x$ \\
         \hline
         0.1 & $-1.6658\times10^{-4}$ & $1.6742\times10^{-4}$ \\
         0.5 & $-2.0574\times10^{-2}$ & $2.3599\times10^{-2}$ \\
         1.0 & $-1.5853\times10^{-1}$ & $5.70796\times10^{-1}$ \\
    \end{tabular}
    \caption{Błędy progresywny i wsteczny dla przybliżenia $\sin(x)\approx x$}
    \label{tab:one_term}
\end{table}

\subsection{Przybliżenie dwoma składnikami ($\sin(x) \approx x - \frac{x^3}{6}$)}

Przyjmujemy $\hat{y} = x - \frac{x^3}{6}$. Otrzymujemy:
\begin{equation}
    \Delta y = \hat{y} - \sin(x), \quad \Delta x = \arcsin(\hat{y}) - x.
\end{equation}
Dla wybranych wartości $x$ otrzymujemy:

\begin{table}[H]
    \centering
    \begin{tabular}{c|cc}
         $x$ & $\Delta y$ & $\Delta x$ \\
         \hline
         0.1 & $8.33\times10^{-8}$ & $-8.37\times10^{-8}$ \\
         0.5 & $2.5887\times10^{-4}$ & $-2.9496\times10^{-4}$ \\
         1.0 & $8.1377\times10^{-3}$ & $-1.4889\times10^{-2}$ \\
    \end{tabular}
    \caption{Błędy progresywny i wsteczny dla przybliżenia $\sin(x)\approx x - \frac{x^3}{6}$}
    \label{tab:two_term}
\end{table}

\section{Zadanie 4}

Zakładamy, że mamy znormalizowany system zmiennoprzecinkowy o parametrach: $\beta=10$, $p=3$, $L=-98$.

\textbf{Poziom niedomiaru (ang. Underflow level - UFL):} 
Najmniejsza dodatnia liczba znormalizowana ma postać
\[
\text{UFL} = 1.00 \times 10^{-98}.
\]

\textbf{Operacja:} Dla $x = 6.87\times10^{-97}$ oraz $y = 6.81\times10^{-97}$:
\[
x - y = (6.87-6.81)\times10^{-97} = 0.06\times10^{-97} = 6.0\times10^{-99}.
\]
Ponieważ $6.0\times10^{-99} < \text{UFL}$, wynik operacji ulega underflow i jest reprezentowany jako $0$.

\end{document}
