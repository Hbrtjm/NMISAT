\documentclass{article}
\usepackage{graphicx}
\usepackage{caption}
\usepackage{booktabs}
\usepackage{amssymb}
\usepackage{amsmath}
\usepackage[dvipsnames]{xcolor}
\usepackage[T1]{fontenc}
\usepackage[utf8]{inputenc}
\usepackage{subcaption} 
\usepackage{float}
\usepackage{geometry}
\geometry{margin=1in}
\usepackage{graphicx}
\usepackage{babel}
\usepackage{animate}
\usepackage{hyphenat}
\usepackage{url} 

\title{Sprawozdanie z laboratorium 3}
\author{Hubert Miklas}
\date{25 Marca 2025}

\begin{document}

\maketitle

\section{Wstęp}
Tematem laboratorium 3. jest analiza metod interpolacji wielomianowej, 
wyrażanie wielomianów metodą Hornera oraz oszacowanie liczby mnożeń 
potrzebnych do ewaluacji wielomianu w różnych reprezentacjach.

\section{Zadanie 1: Obliczenie wielomianu interpolacyjnego}

Mamy dane węzły:
\[
(x_0,y_0)=(-1,2.4), \quad (x_1,y_1)=(1,1.8), \quad (x_2,y_2)=(2,4.5).
\]
Szukamy wielomianu stopnia 2 w postaci:
\[
p(t)=a_2 t^2+a_1 t+a_0.
\]

\subsection{Metoda jednomianów}
Podstawiając poszczególne węzły, otrzymujemy układ równań:
\begin{align}
a_2(-1)^2 + a_1(-1) + a_0 &= 2.4, \label{eq:mon1}\\
a_2(1)^2 + a_1(1) + a_0 &= 1.8, \label{eq:mon2}\\
a_2(2)^2 + a_1(2) + a_0 &= 4.5. \label{eq:mon3}
\end{align}
Co upraszcza się do:
\begin{align*}
a_2 - a_1 + a_0 &= 2.4, \\
a_2 + a_1 + a_0 &= 1.8, \\
4a_2 + 2a_1 + a_0 &= 4.5.
\end{align*}

Odejmując równanie (\ref{eq:mon1}) od (\ref{eq:mon2}):
\[
(a_2 + a_1 + a_0) - (a_2 - a_1 + a_0)= 1.8-2.4 \quad \Longrightarrow \quad 2a_1=-0.6,
\]
stąd:
\[
a_1=-0.3.
\]

Podstawiamy $a_1$ do równania (\ref{eq:mon2}):
\[
a_2 -0.3 + a_0 = 1.8 \quad \Longrightarrow \quad a_2 + a_0 = 2.1. \tag{4}
\]
Natomiast z równania (\ref{eq:mon3}):
\[
4a_2 + 2(-0.3) + a_0 = 4.5 \quad \Longrightarrow \quad 4a_2 -0.6 + a_0 = 4.5,
\]
czyli:
\[
4a_2 + a_0 = 5.1. \tag{5}
\]
Odejmując równanie (4) od (5):
\[
(4a_2 + a_0) - (a_2 + a_0)= 5.1-2.1 \quad \Longrightarrow \quad 3a_2=3,
\]
więc:
\[
a_2=1.
\]
Wstawiając $a_2=1$ do (A):
\[
1 + a_0 = 2.1 \quad \Longrightarrow \quad a_0=1.1.
\]

Stąd wielomian interpolacyjny ma postać:
\begin{equation} \tag{6}
p(t)=t^2-0.3t+1.1.
\end{equation}

\subsection{Metoda wielomianów Lagrange’a}
Wielomian interpolacyjny w postaci Lagrange’a zapisujemy jako:
\[
p(t)=\sum_{k=0}^{2}y_k L_k(t),
\]
gdzie wielomiany bazowe mają postać:
\[
L_k(t)=\prod_{\substack{i=0 \\ i\neq k}}^{2}\frac{t-x_i}{x_k-x_i}.
\]

Obliczamy kolejne $L_k(t)$:
\begin{itemize}
    \item Dla $k=0$, czyli $x_0=-1$:
    \[
    L_0(t)=\frac{(t-x_1)(t-x_2)}{(x_0-x_1)(x_0-x_2)}=\frac{(t-1)(t-2)}{(-1-1)(-1-2)}=\frac{(t-1)(t-2)}{(-2)(-3)}=\frac{(t-1)(t-2)}{6}.
    \]
    \item Dla $k=1$, czyli $x_1=1$:
    \[
    L_1(t)=\frac{(t-x_0)(t-x_2)}{(x_1-x_0)(x_1-x_2)}=\frac{(t+1)(t-2)}{(1+1)(1-2)}=\frac{(t+1)(t-2)}{(2)(-1)}=-\frac{(t+1)(t-2)}{2}.
    \]
    \item Dla $k=2$, czyli $x_2=2$:
    \[
    L_2(t)=\frac{(t-x_0)(t-x_1)}{(x_2-x_0)(x_2-x_1)}=\frac{(t+1)(t-1)}{(2+1)(2-1)}=\frac{(t+1)(t-1)}{3}.
    \]
\end{itemize}

Podstawiając wartości $y_0=2.4$, $y_1=1.8$, $y_2=4.5$, mamy:
\[
p(t)=2.4\cdot\frac{(t-1)(t-2)}{6} -1.8\cdot\frac{(t+1)(t-2)}{2} + 4.5\cdot\frac{(t+1)(t-1)}{3}.
\]

Uprośćmy poszczególne wyrażenia:
\[
\frac{2.4}{6}=0.4,\quad \frac{1.8}{2}=0.9,\quad \frac{4.5}{3}=1.5.
\]
Wówczas:
\[
p(t)=0.4\,(t-1)(t-2) - 0.9\,(t+1)(t-2) + 1.5\,(t+1)(t-1).
\]

Rozwiniemy nawiasy:
\begin{align*}
(t-1)(t-2)&=t^2-3t+2,\\[1mm]
(t+1)(t-2)&=t^2-t-2,\\[1mm]
(t+1)(t-1)&=t^2-1.
\end{align*}

Podstawiając:
\begin{align*}
p(t) &= 0.4\,(t^2-3t+2) - 0.9\,(t^2-t-2) + 1.5\,(t^2-1)\\[1mm]
&= \bigl(0.4t^2-1.2t+0.8\bigr) + \bigl(-0.9t^2+0.9t+1.8\bigr) + \bigl(1.5t^2-1.5\bigr)\\[1mm]
&= (0.4-0.9+1.5)t^2 + (-1.2+0.9)t + (0.8+1.8-1.5)\\[1mm]
&= t^2 - 0.3t + 1.1.
\end{align*}

Stąd:
\begin{equation}    
p(t)=t^2-0.3t+1.1 \tag{7}
\end{equation}
co jest zgodne z wynikiem uzyskanym metodą jednomianów.\\[2mm]

\subsection{Metoda Newtona}
W postaci Newtona wielomian interpolacyjny zapisujemy jako:
\[
p(t)=a_0+a_1(t-x_0)+a_2(t-x_0)(t-x_1),
\]
gdzie współczynniki wyznaczamy w następujący sposób:
\begin{itemize}
    \item $a_0=f(x_0)=2.4$,
    \item $a_1=\dfrac{f(x_1)-f(x_0)}{x_1-x_0}=\dfrac{1.8-2.4}{1-(-1)}=\dfrac{-0.6}{2}=-0.3$,
    \item $a_2=\dfrac{\displaystyle \frac{f(x_2)-f(x_1)}{x_2-x_1}-a_1}{x_2-x_0}=\dfrac{\displaystyle \frac{4.5-1.8}{2-1} - (-0.3)}{2-(-1)}=\dfrac{2.7+0.3}{3}=\dfrac{3.0}{3}=1.$
\end{itemize}

Podstawiając, otrzymujemy:
\[
p(t)=2.4 - 0.3\,(t+1) + (t+1)(t-1).
\]
Ponieważ:
\[
(t+1)(t-1)=t^2-1,
\]
to:
\[
p(t)=2.4-0.3t-0.3 + t^2-1=t^2-0.3t+(2.4-0.3-1)=t^2-0.3t+1.1.
\]
Czyli:
\begin{equation}
p(t)=t^2-0.3t+1.1. \tag{8}
\end{equation}
\subsection{Podsumowanie}
Wszystkie trzy metody – jednomianowa, Lagrange’a oraz Newtona – dają ten sam wielomian interpolacyjny:
\[
p(t)=t^2-0.3t+1.1.
\]
Jest to zgodne z teorią interpolacji, która gwarantuje jednoznaczność wielomianu interpolacyjnego dla danej liczby węzłów \cite{wiki:Interpolacja_wielomianowa}.

\subsection{Kod realizujący wszystkie obliczenia i wyniki}

\begin{verbatim}
    import numpy as np

def mononomial(values):
    n = len(values)
    A = np.array([ [values[j][0] ** i for i in range(n)] for j in range(n) ])
    b = np.array(list(zip(*values))[1])
    coefficients = np.linalg.solve(A,b)
    return np.poly1d(coefficients)

def Lagrange_polynomial(values):
    x_values, y_values = zip(*values)
    n = len(values)
    def L(i, x):
        nonlocal x_values
        numer = np.prod([(x - x_values[j]) for j in range(len(x_values)) if j != i])
        denom = np.prod([(x_values[i] - x_values[j]) for j in range(len(x_values)) if j != i])
        return numer / denom
    
    def P(x):
        return sum(y_values[i] * L(i, x) for i in range(len(x_values)))
    
    return P


def Newton_approximation(values):
    x_values, y_values = zip(*values)
    
    def divided_differences(x_values, y_values):
        n = len(x_values)
        coef = list(y_values)
        for j in range(1, n):
            for i in range(n - 1, j - 1, -1):
                coef[i] = (coef[i] - coef[i - 1]) / (x_values[i] - x_values[i - j])
        return coef
    
    coeffs = divided_differences(x_values, y_values)
    
    def P(x):
        n = len(coeffs)
        result = coeffs[-1]
        for i in range(n - 2, -1, -1):
            result = result * (x - x_values[i]) + coeffs[i]
        return result
    
    return P

def test_interpolation(values):
    poly_mono = mononomial(values)
    poly_lagrange = Lagrange_polynomial(values)
    poly_newton = Newton_approximation(values)
    test_xs = [-1, 1, 2, 0, 3]
    for x in test_xs:
        print(f"x={x}: Monomial={poly_mono(x)}, Lagrange={poly_lagrange(x)}, Newton={poly_newton(x)}")

def main():
    interpolation_points = [(-1,2.4), (1,1.8),(2,4.5)]
    test_interpolation(interpolation_points)

if __name__ == "__main__":
    main()
\end{verbatim}

Dla wybranych wartości $x$, gdzie $x \in {-3,-2,-1, 0, 1, 2, 3} $ prezentują się następująco:

\begin{table}[h]
    \centering
    \begin{tabular}{|c|c|c|c|}
         \hline
         Wartość x & Jednomian & Lagrange & Netwon \\
         \hline
         -3 & 11.8 & 11.0 & 11.0 \\
-2 & 6.0 & 5.699999999999999 & 5.699999999999999 \\
-1 & 2.4 & 2.4 & 2.4 \\
0 & 0.9999999999999999 & 1.1 & 1.1 \\
1 & 1.8000000000000003 & 1.8 & 1.8 \\
2 & 4.800000000000001 & 4.5 & 4.5 \\
3 & 10.000000000000002 & 9.2 & 9.200000000000001 \\
    \hline
    \end{tabular}
    \caption{Porównanie wartości wynikowych dla wybranych x po interpolacji zadaną metodą}
    \label{tab:results}
\end{table}

\section{Zadanie 2: Wyrażenie wielomianu metodą Hornera}
Mamy dany wielomian:
\begin{equation}
    p(t)=3t^3-7t^2+5t-4.
\end{equation}
Celem jest przekształcenie tego wielomianu do postaci Hornera, która pozwala na efektywną ewaluację i redukcję błędów numerycznych (por. \cite{Higham2002,Goldberg1991}).


\subsection{Przekształcenie do postaci Hornera}
Schemat Hornera polega na zagnieżdżonym wyciąganiu wspólnego czynnika \(t\):
\[
p(t)=((a_3t+a_2)t+a_1)t+a_0.
\]
Podstawiając nasze współczynniki, otrzymujemy:
\[
p(t)=((3t-7)t+5)t-4.
\]

\subsection{Weryfikacja poprawności}
Rozwińmy postać Hornera, aby upewnić się, że jest równoważna oryginalnemu wielomianowi:
\begin{align*}
((3t-7)t+5)t-4 &= \bigl((3t^2-7t)+5\bigr)t-4\\[1mm]
&= (3t^3-7t^2+5t)-4\\[1mm]
&= 3t^3-7t^2+5t-4.
\end{align*}
Otrzymany wynik jest identyczny z postacią pierwotną, co potwierdza poprawność przekształcenia.

\subsection{Korzyści płynące z metody Hornera}
Postać Hornera pozwala na:
\begin{itemize}
    \item Redukcję liczby mnożeń – dla wielomianu stopnia 3 potrzebujemy tylko 3 mnożeń i 3 dodawań, zamiast 6 mnożeń i 3 dodawań.
    \item Zmniejszenie błędów zaokrągleń, co jest szczególnie istotne przy ewaluacji wielomianów w arytmetyce zmiennoprzecinkowej (patrz \cite{Higham2002,Goldberg1991,Rycerz}).
\end{itemize}

\section{Zadanie 3: Analiza złożoności obliczeniowej ewaluacji wielomianu}

Rozważmy wielomian \( p(t) \) stopnia \( n-1 \) zapisany w trzech reprezentacjach.

\subsection{Reprezentacja jednomianowa}
W standardowej postaci:
\[
p(t)=a_0+a_1t+a_2t^2+\cdots+a_{n-1}t^{n-1}.
\]
Ewaluację wielomianu można przeprowadzić efektywnie przy użyciu schematu Hornera, który przekształca postać wielomianu do:
\[
p(t)=a_0+t\Bigl(a_1+t\Bigl(a_2+\cdots+t\,a_{n-1}\Bigr)\Bigr).
\]
Schemat Hornera wymaga dokładnie \( n-1 \) mnożeń oraz \( n-1 \) dodawań. Zatem liczba mnożeń wynosi $n-1$ mnożeń. Oznacza to  złożoność czasową względem ilości operacji \( \mathcal{O}(n) \); por. \cite{Higham2002}.

\subsection{Reprezentacja Lagrange’a}
W postaci Lagrange’a wielomian interpolacyjny zapisujemy jako:
\[
p(t)=\sum_{k=0}^{n-1}y_k\,L_k(t),
\]
gdzie wielomian bazowy \( L_k(t) \) ma postać:
\[
L_k(t)=\prod_{\substack{i=0 \\ i\neq k}}^{n-1}\frac{t-x_i}{x_k-x_i}.
\]
Dla każdego \( k \) należy wykonać iloczyn \( n-1 + n-1 \) mnożenia (obliczenie mianownika i licznika, nie licząc w tym dzielenia) – co daje \( 2 n- 2 \) mnożeń na każdy z \( n \) węzłów, czyli łącznie:
\[
2n(n-1)
\]
mnożeń. Dodatkowo, po obliczeniu iloczynów należy pomnożyć każdy \( L_k(t) \) przez \( y_k \) (co daje \( n \) mnożeń) oraz zsumować wszystkie składniki. Łącznie otrzymujemy:
\[
2n(n-1) + n = 2n^2 - n.
\]
jest to rzędu $\mathcal{O}(n^2)$ mnożeń.

\subsection{Reprezentacja Newtona}
W postaci Newtona wielomian interpolacyjny zapisujemy jako:
\[
p(t)=a_0+a_1(t-x_0)+a_2(t-x_0)(t-x_1)+\cdots+a_{n-1}(t-x_0)(t-x_1)\cdots(t-x_{n-2}).
\]
Wyznaczenie tego wielomianu można przeprowadzić również przy użyciu schematu Hornera (adaptowanego do postaci Newtona), co wymaga:
\begin{itemize}
    \item Obliczenia iloczynów w sposób sekwencyjny – przy czym dla wyrazu \( (t-x_0)(t-x_1)\cdots(t-x_{k-1}) \) potrzeba \( k \) mnożeń,
    \item Łączna liczba mnożeń wynosi: \( 0+1+2+\cdots+(n-1)=\frac{n(n-1)}{2} \).
\end{itemize}
Jednak przy zastosowaniu rekurencyjnej formy schematu Hornera dla Newtona (patrz np. \cite{Rycerz}), można obliczyć wartość wielomianu wykonując tylko \( n-1 \) mnożeń. Dlatego przyjmujemy $n-1$ mnożeń.

\subsection{Podsumowanie}
\begin{itemize}
    \item Reprezentacja jednomianowa (przy użyciu schematu Hornera): około \( n-1 \) mnożeń.
    \item Reprezentacja Lagrange’a: około \( n^2 \) mnożeń.
    \item Reprezentacja Newtona (przy użyciu schematu Hornera): około \( n-1 \) mnożeń.
\end{itemize}

Porównanie efektywności ewaluacji wielomianu wskazuje, że metody oparte na schemacie Hornera (jednomiany oraz Newtona) są znacznie bardziej optymalne niż bezpośrednia ewaluacja postaci Lagrange’a. Wynik ten znajduje potwierdzenie w literaturze dotyczącej stabilności i efektywności algorytmów numerycznych \cite{Higham2002,Goldberg1991}.


\section{Zadanie domowe 1 -- Interpolacja funkcji Rungego}
Rozważamy funkcję Rungego:
\begin{equation}
    f(t)=\frac{1}{1+25t^2}.
\end{equation}
Interpolacja tej funkcji przy użyciu równoodległych węzłów na przedziale $[-1,1]$ prowadzi do wystąpienia tzw. \emph{efektu Rungego} – czyli znacznych oscylacji, szczególnie w pobliżu krańców przedziału. Problem ten wynika z właściwości wielomianu interpolacyjnego, który dla dużych stopni może mieć bardzo zmienny przebieg przy równoodległych węzłach (por. \cite{Higham2002}).

\subsection{Węzły równomierne vs. węzły Czebyszewa}
Przy wyborze węzłów równomiernych w przedziale $[-1,1]$, węzły te mają postać:
\[
t_k = -1 + \frac{2k}{n}, \quad k=0,1,\dots,n.
\]
W praktyce, interpolacja funkcji Rungego dla stosunkowo dużej liczby węzłów (np. \( n \ge 10 \)) powoduje wyraźne oscylacje wielomianu interpolacyjnego przy krańcach przedziału.

Lepszym rozwiązaniem jest użycie węzłów Czebyszewa, które są rozmieszczone gęściej przy końcach przedziału, co minimalizuje maksymalny błąd interpolacji. Węzły Czebyszewa dla przedziału \([-1,1]\) definiuje się jako:
\begin{equation}
    t_k = \cos\left(\frac{2k+1}{2n+2}\pi\right), \quad k=0,1,\dots,n.
\end{equation}
Dzięki takiemu rozmieszczeniu błędy interpolacyjne są rozłożone bardziej równomiernie, a oscylacje znacznie się zmniejszają (por. \cite{wiki:Interpolacja_wielomianowa}, \cite{Higham2002}).

\newpage
\section{Zadanie domowe 2 -- Własności wielomianów Legendre’a}

W tym zadaniu rozważamy trzy zagadnienia związane z wielomianami Legendre’a.

\subsection{Ortogonalność}
Wielomiany Legendre’a \( P_n(t) \) są ortogonalne względem miary jednostkowej na przedziale \([-1,1]\), co formalnie zapisujemy jako:
\begin{equation}
    \int_{-1}^{1} P_i(t) P_j(t) \, dt = 0, \quad \text{dla } i \neq j.
\end{equation}
Dowód tej ortogonalności opiera się na własnościach układu wielomianów ortogonalnych i można go znaleźć w literaturze \cite{wiki:Legendre_polynomials}.
\subsection{Rekurencyjna zależność}
Wielomiany Legendre’a spełniają również rekurencyjny wzór, który pozwala na wyznaczanie kolejnych wielomianów:
\begin{equation}
    (n+1) P_{n+1}(t) = (2n+1)t\,P_n(t) - n\,P_{n-1}(t).
\end{equation}
Wzór ten umożliwia generowanie wielomianów \( P_n(t) \) dla \( n\geq 1 \) przy znajomości \( P_0(t) \) i \( P_1(t) \). Standardowe postaci pierwszych kilku wielomianów to:
\[
P_0(t)=1,\quad P_1(t)=t,\quad P_2(t)=\frac{1}{2}(3t^2-1),\quad P_3(t)=\frac{1}{2}(5t^3-3t),\ \ldots
\]

\subsection{Rozkład jednomianów}
Każdy jednomian \( t^k \) (dla \( k=0,1,\dots,6 \)) można wyrazić jako liniową kombinację wielomianów Legendre’a \( \{P_0(t), P_1(t), \dots, P_6(t)\} \). Przykładowo, dla niższych potęg mamy:
\begin{align*}
1 &= P_0(t),\\[1mm]
t &= P_1(t),\\[1mm]
t^2 &= \frac{2}{3}P_2(t) + \frac{1}{3}P_0(t),\\[1mm]
t^3 &= \frac{2}{5}P_3(t) + \frac{3}{5}P_1(t).
\end{align*}
Ogólny proces wyznaczania współczynników opiera się na wykorzystaniu ortogonalności wielomianów Legendre’a – poprzez rzutowanie jednomianu na przestrzeń wielomianów ortogonalnych:
\[
c_k = \frac{\int_{-1}^{1} t^m\,P_k(t)\, dt}{\int_{-1}^{1} \left[P_k(t)\right]^2 dt}.
\]
Dla wyższych potęg (np. \( m=4,5,6 \)) współczynniki można wyznaczyć analogicznie, co daje rozkłady typu:
\begin{align*}
t^4 &= \frac{8}{35}P_4(t) + \frac{4}{7}P_2(t) + \frac{3}{35}P_0(t),\\[1mm]
t^5 &= \frac{8}{63}P_5(t) + \frac{5}{9}P_3(t) + \frac{5}{21}P_1(t),\\[1mm]
t^6 &= \frac{16}{231}P_6(t) + \frac{10}{33}P_4(t) + \frac{5}{11}P_2(t) + \frac{1}{33}P_0(t).
\end{align*}
\newpage
\section{Zadanie domowe 3 -- Interpolacja sklejanymi funkcjami sześciennymi}

Rozważamy interpolację funkcji przy użyciu sklejanych funkcji sześciennych na trzech punktach \( x_0, x_1, x_2 \) o równych odstępach (tj. \( x_1=x_0+h \) oraz \( x_2=x_0+2h \)). Sklejanymi funkcjami sześciennymi (ang. cubic splines) na dwóch przedziałach \([x_0,x_1]\) i \([x_1,x_2]\) są dwie funkcje:
\[
S_0(x)=a_0+b_0(x-x_0)+c_0(x-x_0)^2+d_0(x-x_0)^3,\quad x\in[x_0,x_1],
\]
\[
S_1(x)=a_1+b_1(x-x_1)+c_1(x-x_1)^2+d_1(x-x_1)^3,\quad x\in[x_1,x_2].
\]

\subsection{Warunki interpolacji i ciągłości}
Aby uzyskać gładką funkcję składaną, musimy spełnić następujące warunki:
\begin{enumerate}
    \item \textbf{Interpolacja wartości:}
    \begin{align*}
        S_0(x_0) &= f(x_0),\\[1mm]
        S_0(x_1) &= f(x_1),\\[1mm]
        S_1(x_1) &= f(x_1),\\[1mm]
        S_1(x_2) &= f(x_2).
    \end{align*}
    \item \textbf{Ciągłość pierwszej pochodnej w punkcie \( x_1 \):}
    \[
    S_0'(x_1)=S_1'(x_1).
    \]
    \item \textbf{Ciągłość drugiej pochodnej w punkcie \( x_1 \):}
    \[
    S_0''(x_1)=S_1''(x_1).
    \]
\end{enumerate}

\subsection{Dodatkowe warunki brzegowe}
Aby układ równań był jednoznacznie określony, stosuje się dodatkowo warunki brzegowe. Jednym z popularnych wyborów jest tzw. \emph{natural spline}:
\[
S_0''(x_0)=0 \quad \text{oraz} \quad S_1''(x_2)=0.
\]
W naszym przypadku, mając 3 punkty, mamy 2 segmenty (każdy opisany przez 4 współczynniki, czyli łącznie 8 niewiadomych). Warunki interpolacji (4 równania), ciągłości pierwszej i drugiej pochodnej w punkcie \( x_1 \) (2 równania) oraz dwa warunki brzegowe (2 równania) dają łącznie 8 równań, które pozwalają na wyznaczenie wszystkich współczynników \( a_i, b_i, c_i, d_i \) (dla \( i=0,1 \)).

\subsection{Opis metody rozwiązania}
Rozważamy funkcję określoną w trzech punktach:
\[
x_0,\quad x_1=x_0+h,\quad x_2=x_0+2h,
\]
oraz wartości:
\[
f(x_0)=y_0,\quad f(x_1)=y_1,\quad f(x_2)=y_2.
\]
Celem jest wyznaczenie funkcji sklejanej \(S(x)\) złożonej z dwóch sześciennych segmentów \(S_0(x)\) dla \(x\in[x_0,x_1]\) oraz \(S_1(x)\) dla \(x\in[x_1,x_2]\), przy zachowaniu warunków interpolacyjnych, ciągłości pierwszej i drugiej pochodnej w punkcie \(x_1\) oraz warunków brzegowych typu natural spline:
\[
S_0''(x_0)=0,\quad S_1''(x_2)=0.
\]

\subsection{Definicja funkcji sklejanych}
Przyjmujemy następujące postaci segmentów:
\[
S_0(x)=a_0+b_0(x-x_0)+c_0(x-x_0)^2+d_0(x-x_0)^3,\quad x\in[x_0,x_1],
\]
\[
S_1(x)=a_1+b_1(x-x_1)+c_1(x-x_1)^2+d_1(x-x_1)^3,\quad x\in[x_1,x_2].
\]

\subsection{Warunki interpolacyjne}
\begin{enumerate}
    \item \(S_0(x_0)=y_0\) $\Longrightarrow$ \(a_0=y_0.\)
    \item \(S_0(x_1)=y_1\) $\Longrightarrow$
    \[
    a_0+b_0h+c_0h^2+d_0h^3=y_1.
    \]
    \item \(S_1(x_1)=y_1\) $\Longrightarrow$ \(a_1=y_1.\)
    \item \(S_1(x_2)=y_2\) $\Longrightarrow$
    \[
    a_1+b_1h+c_1h^2+d_1h^3=y_2.
    \]
\end{enumerate}

\subsection{Warunki ciągłości}
\begin{enumerate}
    \item \textbf{Ciągłość pierwszej pochodnej w \(x_1\):}
    \[
    S_0'(x)=b_0+2c_0(x-x_0)+3d_0(x-x_0)^2,
    \]
    więc
    \[
    S_0'(x_1)=b_0+2c_0h+3d_0h^2.
    \]
    Dla \(S_1(x)\):
    \[
    S_1'(x)=b_1+2c_1(x-x_1)+3d_1(x-x_1)^2,
    \]
    a zatem
    \[
    S_1'(x_1)=b_1.
    \]
    Warunek ciągłości pierwszej pochodnej:
    \[
    b_0+2c_0h+3d_0h^2=b_1.
    \]
    
    \item \textbf{Ciągłość drugiej pochodnej w \(x_1\):}
    \[
    S_0''(x)=2c_0+6d_0(x-x_0) \quad \Longrightarrow \quad S_0''(x_1)=2c_0+6d_0h.
    \]
    \[
    S_1''(x)=2c_1+6d_1(x-x_1) \quad \Longrightarrow \quad S_1''(x_1)=2c_1.
    \]
    Warunek ciągłości drugiej pochodnej:
    \[
    2c_0+6d_0h=2c_1 \quad \Longrightarrow \quad c_1=c_0+3d_0h.
    \]
\end{enumerate}

\subsubsection*{Warunki brzegowe (natural spline)}
\begin{enumerate}
    \item \(S_0''(x_0)=0\) $\Longrightarrow$ \(2c_0=0\) $\Longrightarrow$ \(c_0=0.\)
    \item \(S_1''(x_2)=0\). Skoro \(x_2=x_1+h\), mamy:
    \[
    S_1''(x_2)=2c_1+6d_1h=0 \quad \Longrightarrow \quad c_1=-3d_1h.
    \]
\end{enumerate}

\subsection{Wyznaczenie współczynników}
Zapiszemy teraz kolejne równania i wyznaczymy współczynniki.

\subsection*{Segment \(S_0(x)\) na \([x_0,x_1]\)}
\begin{enumerate}
    \item \(a_0=y_0.\)
    \item \(S_0(x_1)=y_1\) daje:
    \[
    y_0+b_0h+0\cdot h^2+d_0h^3=y_1 \quad \Longrightarrow \quad b_0h+d_0h^3=y_1-y_0.
    \]
    Stąd:
    \[
    b_0=\frac{y_1-y_0}{h}-d_0h^2.
    \]
\end{enumerate}

\subsection*{Segment \(S_1(x)\) na \([x_1,x_2]\)}
\begin{enumerate}
    \item \(a_1=y_1.\)
    \item \(S_1(x_2)=y_2\) daje:
    \[
    y_1+b_1h+c_1h^2+d_1h^3=y_2 \quad \Longrightarrow \quad b_1h+c_1h^2+d_1h^3=y_2-y_1.
    \]
\end{enumerate}

\subsection*{Ciągłość pochodnej w \(x_1\)}
\[
b_0+2c_0h+3d_0h^2=b_1.
\]
Ponieważ \(c_0=0\), mamy:
\[
b_0+3d_0h^2=b_1.
\]
Podstawiając \(b_0\):
\[
\frac{y_1-y_0}{h}-d_0h^2+3d_0h^2=\frac{y_1-y_0}{h}+2d_0h^2=b_1.
\]

\subsection*{Ciągłość drugiej pochodnej w \(x_1\)}
\[
c_1=c_0+3d_0h \quad \Longrightarrow \quad c_1=3d_0h,
\]
oraz z warunku brzegowego dla \(S_1''(x_2)\):
\[
2c_1+6d_1h=0 \quad \Longrightarrow \quad c_1=-3d_1h.
\]
Zatem:
\[
3d_0h=-3d_1h \quad \Longrightarrow \quad d_0=-d_1.
\]

\subsection{Wyznaczenie \(d_0\)}
Podstawiamy teraz do równania \(S_1(x_2)=y_2\). Korzystamy z faktu, że:
\[
b_1=\frac{y_1-y_0}{h}+2d_0h^2,
\]
a \(c_1=3d_0h\). Wówczas równanie \(S_1(x_2)=y_2\) przyjmuje postać:
\[
y_1+\left(\frac{y_1-y_0}{h}+2d_0h^2\right)h+3d_0h\cdot h^2+d_1h^3=y_2.
\]
Ponieważ \(d_1=-d_0\), mamy:
\[
y_1+\frac{y_1-y_0}{h}\cdot h+2d_0h^3+3d_0h^3-d_0h^3=y_2.
\]
Upraszczając:
\[
y_1+(y_1-y_0)+4d_0h^3=y_2.
\]
Stąd:
\[
4d_0h^3=y_2-2y_1+y_0 \quad \Longrightarrow \quad d_0=\frac{y_2-2y_1+y_0}{4h^3}.
\]
A zatem:
\[
d_1=-\frac{y_2-2y_1+y_0}{4h^3}.
\]

\subsection{Wyznaczenie \(b_0\) i \(b_1\)}
Mamy:
\[
b_0=\frac{y_1-y_0}{h}-d_0h^2=\frac{y_1-y_0}{h}-\frac{y_2-2y_1+y_0}{4h^3}\cdot h^2
=\frac{y_1-y_0}{h}-\frac{y_2-2y_1+y_0}{4h}.
\]
Natomiast:
\[
b_1=b_0+3d_0h^2=\frac{y_1-y_0}{h}-\frac{y_2-2y_1+y_0}{4h}+3\frac{y_2-2y_1+y_0}{4h}
=\frac{y_1-y_0}{h}+\frac{2(y_2-2y_1+y_0)}{4h}.
\]
Uproszczając:
\[
b_1=\frac{y_1-y_0}{h}+\frac{y_2-2y_1+y_0}{2h}.
\]

\subsection{Wyznaczenie \(c_1\)}
Z ciągłości drugiej pochodnej:
\[
c_1=3d_0h=\frac{3h\,(y_2-2y_1+y_0)}{4h^3}=\frac{3(y_2-2y_1+y_0)}{4h^2}.
\]

\subsection{Ostateczne współczynniki funkcji sklejanych}
\subsubsection*{Dla przedziału \([x_0,x_1]\) (funkcja \(S_0(x)\))}
\[
\begin{aligned}
a_0 &= y_0,\\[1mm]
b_0 &= \frac{y_1-y_0}{h}-\frac{y_2-2y_1+y_0}{4h},\\[1mm]
c_0 &= 0,\\[1mm]
d_0 &= \frac{y_2-2y_1+y_0}{4h^3}.
\end{aligned}
\]

\subsubsection*{Dla przedziału \([x_1,x_2]\) (funkcja \(S_1(x)\))}
\[
\begin{aligned}
a_1 &= y_1,\\[1mm]
b_1 &= \frac{y_1-y_0}{h}+\frac{y_2-2y_1+y_0}{2h},\\[1mm]
c_1 &= \frac{3(y_2-2y_1+y_0)}{4h^2},\\[1mm]
d_1 &= -\frac{y_2-2y_1+y_0}{4h^3}.
\end{aligned}
\]

\subsection{Podsumowanie}
Ostatecznie funkcja sklejana \(S(x)\) jest dana wzorami:
\[
S(x)=
\begin{cases}
S_0(x)=y_0+\left[\frac{y_1-y_0}{h}-\frac{y_2-2y_1+y_0}{4h}\right](x-x_0)
+\frac{y_2-2y_1+y_0}{4h^3}(x-x_0)^3, & x\in[x_0,x_1],\\[2mm]
S_1(x)=y_1+\left[\frac{y_1-y_0}{h}+\frac{y_2-2y_1+y_0}{2h}\right](x-x_1)
+\frac{3(y_2-2y_1+y_0)}{4h^2}(x-x_1)^2
-\frac{y_2-2y_1+y_0}{4h^3}(x-x_1)^3, & x\in[x_1,x_2].
\end{cases}
\]
Spełnia ona warunki interpolacji, ciągłości pierwszej i drugiej pochodnej w punkcie \(x_1\) oraz warunki naturalne na brzegach, co gwarantuje gładkość przejścia między segmentami.



\bibliographystyle{plain}
\bibliography{references}

\end{document}
